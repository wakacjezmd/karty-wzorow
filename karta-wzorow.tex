\documentclass[10pt,a4paper,twocolumn]{article}

\usepackage{multicol}
\usepackage{amsbsy, amssymb, latexsym, amsmath, braket}
\usepackage[tiny]{titlesec}
\usepackage[hmargin=0.5cm,vmargin=1.0cm]{geometry}
\usepackage[utf8x]{inputenc}
\usepackage{polski}
\usepackage{scalefnt}
\usepackage[yyyymmdd,hhmmss]{datetime}
\usepackage{commath}

% Potrzebne do algorytmu Euklidesa.
\usepackage{tikz}
\usetikzlibrary{tikzmark}

\newcommand{\angles}[1]{\left\langle #1 \right\rangle}


\newcommand{\entry}{$\bullet$\hspace{0.15em}}
\newcommand{\subentry}{$\circledcirc$\hspace{0.15em}}
% https://tex.stackexchange.com/a/7045/80219
\newcommand{\textsubentry}[1]{\tikz[baseline=(char.base)]{
            \node[shape=circle,draw,inner sep=1pt] (char) {#1};\hspace{0.15em}}}

% Automatyczne generowanie listy liczby pierwszych:
% https://tex.stackexchange.com/a/134320/80219
\makeatletter
\def\primes#1#2{{%
  \def\comma{\def\comma{, }}%
  \count@\@ne\@tempcntb#2\relax\@curtab#1\relax
  \@primes}}
\def\@primes{\loop\advance\count@\@ne
\expandafter\ifx\csname p-\the\count@\endcsname\relax
\ifnum\@tempcntb<\count@\else
  \ifnum\count@<\@curtab\else\comma\the\count@\fi\fi\else\repeat
\@tempcnta\count@\loop\advance\@tempcnta\count@
\expandafter\let\csname p-\the\@tempcnta\endcsname\@ne
\ifnum\@tempcnta<\@tempcntb\repeat
\ifnum\@tempcntb>\count@\expandafter\@primes\fi}
\makeatother

\titlespacing{\section}{0pt}{0pt}{0pt}
\titlespacing{\subsection}{0pt}{0pt}{0pt}
\titlespacing{\subsubsection}{0pt}{0pt}{0pt}

% Wyłącz numerowanie stron.
\pagenumbering{gobble}

\setlength{\parindent}{0pt}
% Odległość pomiędzy liniami. Zmniejsz, jeżeli brakuje miejsca.
\setlength{\parskip}{0.5ex}

\title{Karta wzorów z matematyki dyskretnej}

\begin{document}
% Rozmiar czcionki.
\scalefont{.8}

\text{\tiny{
    Wersja z \today\ o \currenttime\ (\pdfmdfivesum file{./karta-wzorow.tex})
}}

\section{Sumy}

${S_n = \sum^{n}_{k = 0} a \cdot x^k} = {a \frac{1 - x^{n+1}}{1 - x}}$;

$\sum_{i=0}^{n-1}\frac{1}{(i+1)(i+2)} = \frac{n}{n+1}$;

$\sum_{i=0}^{n}(-1)^i i^2 = (-1)^n\frac{n(n+1)}{2}$;

\subsection{Sumowanie przez części}

${\Delta f(x) = f(x + 1) - f(x)}$;
${\Delta x^{\underline{n}}} = {n x^{\underline{n - 1}}}$;
${\mathcal{S}^{b}_{a} f}={g |^b_a} = {\sum^{b - 1}_{k = a} f(k)}$;
${\Delta (rt)} = {r \Delta t + E t \Delta r \text{, gdzie } E t(x)} =
  {t(x + 1)}$;
${\mathcal{S}^{b}_{a} r \Delta t}= {r t |^b_a - \mathcal{S}^b_a E t \Delta r}$;

\subsection{Dwumian}

${\binom{n}{k} = {\frac{n^{\underline{k}}}{k!}}}$;
${\sum_{i = 0}^n \binom{n}{i} = 2^n}$;
${\sum_{i = 0}^{n} (-1)^i \binom{n}{i} = [ n = 0 ]}$;
$\binom{n}{k} = \binom{n}{n - k}$;

$\binom{n}{k} = \frac{n}{k} \binom{n - 1}{k - 1}$;

${(x + y)^n = \sum_{i \geq 0} \binom{n}{i} x^i y^{n-i}, n \in \mathbb{N}}$;

${(1 + x)^r = \sum_{i \geq 0} \binom{r}{i} x^i, r \in \mathbb{R}, |x| < 1}$;

$(x + y + z)^n =
  \sum_{0 \leq a, b \leq n, a+b \leq n} \binom{n}{a,b} x^a y^b z^{n-a-b}$;

$\binom{n}{a,b} = \binom{n}{a}\binom{n-a}{b} = \frac{n!}{a! b! (n - a - b)!}$;

$\binom{n}{k}\binom{k}{i} = \binom{n}{i}\binom{n-i}{k-i}$;

$\binom{n}{k} = \begin{cases}
    1, k = 0 \text{ lub } k = n \\
    \binom{n - 1}{k} + \binom{n - 1}{k - 1}, 0 < k < n
\end{cases}
$;

% Tożsamość Cauchy'ego.
Toż. (Cauchy): $\sum^k_{i=0}\binom{n}{i}\binom{m}{k - i} = \binom{n + m}{k}$;

$\text{$\sum$ równoległe: } \sum^k_{i=0} \binom{y + i}{i} =
  \binom{y + k + 1}{k}$;

$(-1)^i\binom{x}{i} = \binom{i - 1 - x}{i} \text{, bo }  x^{\underline{i}} =
  (-1)^i(i - 1 - x)^{\underline{i}}$;

$a_n = \sum_i\binom{n}{i}(-1)^i b_i \iff b_n = \dots$;

% Inwersje.
\subsection{Inwersje}

$\Pi = 1^{\lambda_1}2^{\lambda_2}\dots n^{\lambda_n}$, to $sgn(\Pi) =
  (-1)^{\Sigma_i\lambda_{2i}}$;

% Liczby Stirlinga I rodzaju.
\subsection{Liczby Stirlinga}

${n \brack k}$ - podziały $n$-zbioru na $k$ cykli (kolejność w cyklu istotna z
  dokładnością do cyklicznego przesunięcia);

${n \brack 0} = [ n = 0 ]$;
${n \brack 1} = (n - 1)!, n > 0$;
$\sum_k{n \brack k} = n!$;

${n \brack k} = (n-1){n-1 \brack k}+{n-1 \brack k-1}$ dla ${k > 0}$;

% Liczby Stirlinga II rodzaju.
${n \brace k}$ - podziały $n$-zbioru na $k$ bloków;

${n \brace 0} = [n = 0]$;
${n \brace 1} = 1$;
${n \brace 2} = 2^{n-1}-1$;

${n \brace k} = k{n-1 \brace k} + {n-1 \brace k-1}$ dla ${k > 0}$;

% Własności liczb Stirlinga.
$x^n = \sum_k{n \brace k} x^{\underline{k}}$;

$x^{\overline{n}} = \sum_k{n \brack k}x^k$;

$(-x)^{\overline{k}} = (-1)^kx^{\overline{k}}$;

$x^n = \sum_k {n \brace k} (-1)^{n-k} x^{\overline{k}}$;

$x^{\underline{n}} = \sum_k {n \brack k} (-1)^{n-k} x^{k}$;

$x^n = \sum_{i,k} {n \brace i}{i \brack k}(-1)^{n-i}x^k$;

$\sum_i{n \brace i}{i \brack k}(-1)^{n-i} = [n=k] =
  \sum_i{n \brack i}{i \brace k}(-1)^{n-i}$;

$a_n = \sum_i{n \brace i}(-1)^ib_i \iff b_n = \sum_i{n \brack i}(-1)^ia_i$;

% Funkcje tworzące.
\section{Funkcje tworzące}

\entry
$\alpha A(x) + \beta B(x) \leftrightsquigarrow \angles{\alpha a_n + \beta b_n}$;

\entry
$x^mA(x) \leftrightsquigarrow
  \angles{\underbrace{0, \dots, 0}_m, a_0, a_1, \dots}$;

\entry
$\frac{A(x) - \sum_{i=0}^{m-1}a_ix^i}{x^m} \leftrightsquigarrow
  \angles{a_m, a_{m+1}, \dots}$;

\entry
$A(x) \cdot B(x) =
  \sum_n \sum_k a_kb_{n-k}x^n \leftrightsquigarrow
  \angles{\sum_k a_k b_{n-k}}_n$;

\entry
$A'(x) = a_1 + 2a_2x + \dots \leftrightsquigarrow = \angles{(n+1)a_{n+1}}_n$;

\entry
$\int_0^xA(t)dt \leftrightsquigarrow \angles{\frac{a_{n-1}}{n}}_{n \geq 1}$;

\subsection{Przykłady funkcji tworzących}

\entry
$\angles{1,1,\dots} \leftrightsquigarrow \frac{1}{1-x}$;

\entry
$\angles{a_n} \cdot \angles{1,1,\dots} = \angles{a_0 + \dots + a_n}_n$;

\entry
$\frac{1}{(1-x)^k} \leftrightsquigarrow \angles{\binom{n + k -1}{k - 1}}_n$;

\entry
$ln \frac{1}{1-x} \leftrightsquigarrow
  \angles{0, 1, \frac{1}{2}, \frac{1}{3}, \dots}$;

\subsection{Wykładnicze funkcje tworzące}

\entry
$B_e(x) = \sum_n \frac{b_nx^n}{n!}$;

\entry
$A_e(x) \cdot B_e(x) =
  \sum \left ( \sum_k \binom{n}{k} a_k b_{n-k} \right ) x^n / n!$;

\entry
$B'e(x) = \sum_n b_{n+1} x^n/n!$;

\entry
$\int^x_0 \sum_{n \geq 0} b_nt^ndt = \sum_{n \geq 1} b_{n-1}x^n/n!$;

\entry
$\frac{C}{(1-\lambda x)^k} =
  \sum_{n\geq 0} C \binom{n + k - 1}{k - 1} \lambda^n x^n$;

% TODO: Dodaj przykład rozwiązania równania rekurencyjnego przy pomocy f.t.

\entry
Proste ciągi i ich funkcje tworzące:
\begin{tabular}{ | l | l | l |  }
    \hline
    $<1,0,0,\dots>$
      & $\sum_{n \geq 0}[n=0] z^n$
      & $1$ \\
    $<0,\dots, 0,1,0,\dots>$
      & $\sum_{n \geq 0}[n=m] z^n$
      & $z^m$ \\
    $<1,1,1,1>$
      & $\sum_{n \geq 0} z^n$
      & $\frac{1}{1-z}$ \\
    $<1,-1,1,-1>$
      & $\sum_{n \geq 0} (-z)^n$
      & $\frac{1}{1+z}$ \\
    co $m$-te
      & $\sum_{n \geq 0} [m | n]z^n$
      & $\frac{1}{1-z^m}$ \\
    $<1,2,3,4>$
      & $\sum_{n \geq 0} (n+1)z^n$
      & $\frac{1}{(1-z)^2}$ \\
    $<1,c,\frac{c}{2},\frac{c}{2}>$
      & $\sum_{n \geq 0} \binom{c}{n}z^n$
      & $(1+z)^c$ \\
    $<1,c,c^2,c^3>$
      & $\sum_{n \geq 0} c^n z^n$
      & $\frac{1}{1-cz}$ \\
    $<1,\frac{m+1}{m},\frac{m+2}{m},>$
      & $\sum_{n \geq 0} \binom{m+n}{m} z^n$
      & $ \frac{1}{(1-z)^{m+1}}$ \\
    $<0,1,\frac{1}{2},\frac{1}{3}>$
      & $n \geq 1$
      & $\ln \frac{1}{(1-z)}$ \\
    \hline
\end{tabular}

\section{Magiczne liczby}

\begin{tiny}
  \primes{1}{2719}
\end{tiny}

\subsection{Liczby Catalana}

\entry
$C_n$ - liczba nawiasowań w wyr. $x_0 \cdot x_1 \cdot \dots \cdot x_n$;

\entry
$C_n = \sum_kC_kC_{n-1-k} + [n=0]$,
  skąd $C(x) = \sum_n C_nx^n = x(C(x))^2 + 1, C(0) = 1$,
  więc $C(x) = \frac{1 - \sqrt{1-4x}}{2x}$, a zatem $C(x) =
  \sum_{k\geq 0} \frac{1}{k+1}\binom{2k}{k}x^k$;

\subsection{Liczby Bella}

\entry
${B_n = \sum_k{n \brace k}}$ - łączna l. podziałów $n$-zbioru na bloki;

\entry
${B_{n+1} = \sum_k \binom{n}{k} B_k}$, skąd $B'(x) = e^xB(x), B(0) = 1$,
  gdzie $B(x) = \sum_n B_nx^n/n!$. Całkujemy i dostajemy $ln B = e^x + c$,
  skąd $B(x) = e^{e^x-1} =
  \sum_{n=0}^\infty \left ( {1 \over e} \sum_{k=0}^\infty {k^n \over k!}\right)
  {x^n \over n!}$;

% Enumeratory.
\subsection{Enumeratory kombinacji}

\entry
$(1+t)\cdot\overset{n}{\ldots}\cdot(1+t) = \sum_r\binom{n}{r}t^r$;

\entry
Kombinacja z dowolnymi powt.: $(1+t+t^2+\dots)^n=(1-t)^{-n}=
  \sum_r\binom{-n}{r}{-t}^r=\sum_r\binom{n+r-1}{r}t^r$;

\entry
Każdy elem. co najmniej raz: $(t+t^2+\dots)^n = t^n\sum_r {n+r-1 \over r}t^r =
  \sum^\infty_{r=n}\binom{r-1}{n-1}t^r$;

\subsection{Enumeratory permutacji}

\entry
$r$-perm. bez powt.: $(1+t)^n = \sum_rn^{\underline{r}}{t^r \over r!}$;

\entry
Dow. l. powt.: ${(1+t+t^2/2! + \dots)^n} = {(e^{t})^n} =
  {\sum_rn^r{t^r \over r!}}$;

\entry
Każdy elem. co najmniej raz: $(t + t^2/2! + \dots)^n = (e^t - 1)^n =
  \sum^\infty_{r=0}\frac{t^r}{r!}\sum^n_{j=0}\binom{n}{j}(-1)^j(n-j)^r$;

\entry
Ciąg $A$, $B$, $C$ długości $r$ t., że $\#A>0$, $2\mid\#B$. Traktuj jako r-permu.
  z powt.:
  $\overbrace{(e^t - 1)}^A\cdot\overbrace{((e^t - e^{-t})/2)}^B
  \cdot\overbrace{e^t}^C =
  \sum_{r \geq 1} {1 \over 2}(3^r - 2^r - 1){t^r \over r!}$;

  \entry 
Liczba ścieżek w prostokącie $n \times m$: $\binom{n+m}{n}$

\section{Zasada włączania-wyłączania}

\entry
Liczba elem. o $j$ własn.: $S_j =
  \sum_{1 \leq i_1 < \dots < i_j \leq n} |A_{i_1} \cap \dots \cap A_{i_j}|$;

\entry
Liczba elem. o dokładnie $k$ własn.:
  $D(k) = \sum_{j\geq k} \binom{j}{k} (-1)^{j-k}S_j$;

\entry
Zliczanie $n$-nieporządków: $A_i = \left\{n\text{-perm.} | f(i) = i \right\}$.
  Wtedy $|A_{i_1} \cap \dots \cap A_{i_j}| = (n-j)!$, zatem
  $D(0) = \sum^n_j(-1)^j\binom{n}{j}(n-j)!=n!\sum^n_j(-1)^j{1 \over j!}$;

% TODO: Wieżomiany.
\section{Podział liczby / enumerator podziałów}

Podział liczby $n$ to przedstawienie $n$ jako sumy nierosnących dod. skład.;

L. podz. $n$ na $\leq k$ skład. $\leftrightsquigarrow$
  l. podz. $n+k$ na dokł. $k$ skład.;

Enum. podziałów $P(x) = \sum_nP_nx^n =
  (1+x+x^2+\dots)\cdot(1+x^2+x^4+\dots)\dots(1+x^k+\dots)\dots =
  \frac{1}{(1-x)(1-x^2)\dots(1-x^k)\dots}$, gdzie wybranie z $k$-tego nawiasu
  $x^{ik} \leftrightsquigarrow$ wzięciu do podziału $i$ razy składnika $k$;

Enum. podz. o składnikach $\leq k$: $p_{\leq k} (x) =
  \frac{1}{(1-x)(1-x^2)\dots(1-x^k)}$;

Enum. podz. 1 zł na grosze:
  $\left [ x^{100} \right ]
  \frac{1}{(1-x)(1-x^2)(1-x^5)(1-x^{10})(1-x^{20})(1-x^{50})}$;

Enum. podz. na składniki $>1$: $\frac{1}{(1-x^2)(1-x^3)\dots}=(1-x)P(x) =
  P_n - P_{n-1}, n>0$;

Enum. podz. o różnych s.: $r(x) = (1+x)(1+x^2)(1+x^3)\dots$;

Enum. podz. o nieparz. s.: $n(x)=\frac{1}{(1-x)(1-x^3)\dots}$;

Tw.: $\#$ podz. $n$ na różne s. $= \#$ podz. na nieparz. s.: $n(x) = r(x)$;

% TODO: Przykład użycia.
Toż. Eulera: $nP_n = \sum_{k=0}^{n-1}\sigma(n-k)P_k$,
  gdzie $\sigma = \sum_{k|n}k$;

\section{Grafy}

Lem. o uściskach dł.: $\sum_{v\in V} deg(v) = 2|E|$;

$H$ podgrafem indukowanym $G
  \iff \forall_{u,v \in V[H]} \set{u, v} \in E[G] \implies \set{u, v}\in E[H]$;

Droga nie powtarza krawędzi, a ścieżka wierzchołków;

$G$ spójny $\iff \forall_{u,v\in V[G]} \exists e_{uv}$;

$G$ $k$-reguralny $\iff \forall_v deg(v) = k$;

$G$ dwudzielny, gdy $V[G] = V_1 \cup V_2, V_1 \cap V_2 = \emptyset$ i każda
  krawędź ma jeden koniec w $V_1$, a drugi w $V_2$;

$K_{|V|, |U|}$ pełny dwudzielny, gdy $E=\set{\set{v, u}, v \in V, u \in U}$;

$v$ rozcinający, gdy usunięcie $v$ zwiększa l. spójnych s.;

Tw.: $G$ dwudzielny $\iff$ nie zawiera cykli nieparz. dł.;

\subsection{Cykle}

C. E. --- krawędzie; C. H. --- wierzchołki; Graf h. --- graf ze ścieżką H.;

Tw. Eulera: $G$ ma c. E. $\iff \forall_{v \in V[G]} 2|deg(v)$;

Tw.: Silnie spójny $G$ ma skierowany c. E.
  $\iff \forall_{v \in V[G]} deg_{in}(v) = deg_{out}(v)$;

$G$ ma c. H., to po usunięciu dow. k wierz. rozpada się na co najw.
  k spójnych s.;

$G = \angles{V, U; E}$ dwudzielny ma c. H., to $|V|=|U|$;

Tw.: Każdy turniej jest półhamilton.;

Tw.: Turniej ma c. H. $\iff$ jest silnie spójny;

Tw.: Turniej spójny, to ma c. H.;

Tw. (Ore): $n=|V|\geq 3$ i
  $\forall_{\set{v, w} \not\in E} deg(v) + deg(w) \geq n$, to $G$ ma c. H.;

\subsection{Drzewa}

Tw. (Cayley): Jest $n^{n-2}$ etykietowanych drzew $n$-wierzchołkowych;

\subsection{Planarność}

Wz. Eulera: $n - m + f = 2$, gdzie $m = |E[G]|$;

Tw.: W grafie plan. z $n \geq 3$ mamy $m \leq 3n -6$;

Tw. Kuratowskiego: $G$ nieplan. $\iff G$ zawiera podgraf homeomorficzny
  z $K_{3,3}$ lub z $K_5$ (homeomorficzny, czyli
  izomorficzny po ew. dołożeniu wierzchołków na krawędziach);

$G$ planarny $\implies \exists_{v \in G[V]} deg(v) \leq 5$;

\subsection{Kolorowanie wierzchołków}

Kolorowanie $G$ za pomocą $k$ kolorów, to
  $f: V[G] \rightarrow \set{1, \dots, k}$ t., że $f(u) \neq f(v)$ dla
  $\set{u, v} \in E[G]$. Najm. k t., że $\exists k$-kolorowanie $G$, to liczba
  chromatyczna $\chi (G)$.

$\chi(G) \leq 2 \Leftrightarrow G$ dwudzielny;

$\chi(G) \leq k \Leftrightarrow \chi(G) \leq k$ dla każdej dwuspójnej s. $B$
  grafu $G$;

Tw. o 4 barwach: $G$ plan. $\implies\chi(G)\leq 4$;

Tw. Brooksa: $G$ spójny, nie cykl nieparz. dł., nie klika, to
  $\chi(G) \leq \Delta$, gdzie $\Delta$, to maks. stop. wierz. w $G$.;

Tw.: $\chi(G) \leq \Delta + 1$;

$f_G(t)$ --- wielomian chrom. (liczba kolorowań $G$ za pomocą $t$ kolorów);

$f_{\overline{K_n}}(t) = t^n$;
$f_{K_n} = t^{\underline{n}}$;

Tw.: $e = \set{v, w} \not\in E[G]$, to $f_G(t)=f_{G\cup e}(t) + f_{G/e}(t)$.

\subsection{Kolorowanie krawędzi}

Funkcja $f: E[G] \rightarrow \set{1, \dots, k}$, to kolorowanie krawędziowe,
  jeśli kraw. incydentne mają różne kolory. Indeks chromatyczny $\chi_e(G)$, to
  najmniejsze $k$, dla którego istnieje $k$-kolorowanie kraw.

Tw. Vizinga: $\forall_G \chi_e(G) \leq \Delta(G) +1$;

Tw. (K{\"o}nig): $G$ dwudzielny, to $\chi_e(G) = \Delta(G)$;

\subsection{Systemy różnych reprezentantów}

SSR dla rodziny zbiorów $\angles{A_i}_{i\in I},$ to ciąg elem.
  $\angles{a_i}_{i\in I}$ t., że
  $\forall_{i\in I} a_i \in A_i$ oraz $a_i \neq a_j$
  (skojarzenia w g. dwudzielnym);

Tw. (Hall): SRR dla skończonej r. zb. skończonych $\angles{A_i}_{i=1}^n,$
  istnieje
  $\iff \forall_{J\subseteq\set{1,\dots,n}} |\bigcup_{j\in J}A_j| \geq |J|$;

$G$ dwudzielny $r$-regularny $\Rightarrow r$-kolorowalny kraw.;

$G$ dwudzielny, regularny ma pełne skojarzenie;

$G$ $(n-m)$-regularny $\Rightarrow \exists$ pełne skojarzenie;

Tw.: Podziały $\mathcal{A}$ i $\mathcal{B}$ mają wspólny SRR $\Leftrightarrow
  \forall_{J\subseteq I} |\bigcup_{j\in J}g(A_j)| \geq |J|$, gdzie
  $g(C) = \set{j | C \cap B_j \neq \emptyset}$;

$A_1 \cup \dots \cup A_n = B_1 \cup \dots \cup B_n$ i
  $\forall_{1 \leq i \leq n}|A_i| = |B_i| =r \Rightarrow \mathcal{A}$ i
  $\mathcal{B}$ mają SRR;

\section{Teoria liczb}

% NWD.
\entry
$NWD(a,b) = min_{+}\set{ax+by | x,y\in \mathbb{Z}}$;
\entry
$a \perp b \Leftrightarrow NWD(a,b)=1$;

% Algorytm Euklidesa.
\entry
Alg. Euklidesa: $NWD(a,b)=ax+by$. Jeśli $b=0$, to
  $\angles{x,y} \angles{1,0}$, wpp
  $NWD(a,b) = NWD(b, a \mod b) = bx' + (a \mod b)y'$
  ($\angles{x,y} \leftarrow \angles{y', x' - y'\cdot\lfloor{a/b}\rfloor}$);

% Tabelka z przykładem działania algorytmu Euklidesa.
% TODO: Dodaj ramkę.
\entry
\begin{tabular}{c c c c}
    20& 56 & 3\tikzmark{aedst3} &-1 \\
    16& 20 & -1\tikzmark{aedst2} &\tikzmark{aesrc3}1 \\
    4 & 16 & 1\tikzmark{aedst1} &\tikzmark{aesrc2}0 \\
    0\tikzmark{aesrc0} & 4 &\tikzmark{aedst0}0 & \tikzmark{aesrc1}1 \\
\end{tabular}
\begin{tikzpicture}[overlay, remember picture, yshift=.25\baselineskip,
    shorten >=.5pt, shorten <=.5pt]
    \draw [->] ({pic cs:aesrc0}) [bend left] to ({pic cs:aedst0});
    \draw [->] ([yshift=.75pt]{pic cs:aesrc1}) -- ({pic cs:aedst1});
    \draw [->] ([yshift=.75pt]{pic cs:aesrc2}) -- ({pic cs:aedst2});
    \draw [->] ([yshift=.75pt]{pic cs:aesrc3}) -- ({pic cs:aedst3});
\end{tikzpicture}

% Podstawowe wiadomości z teorii liczb.
\entry
$a = \prod^m_{i=1}p_i$;

\entry
${a | bc \land a \perp b} \Rightarrow {a | c}$

\entry
${NWW(a,b)} = {ab/NWD(a,b)} = {\prod^k_{i=1}p_i^{max(\alpha_i, \beta_i)}}$;

\entry
${a \equiv_{n} b}$ oraz ${c \equiv_{n} d} \implies
  {a+c\equiv_n b+d}$ oraz ${a\cdot c \equiv_n b\cdot d}$;

\entry
${d \perp n} \land {ad \equiv bd \pmod{n}} \Rightarrow {a \equiv b \pmod{n}}$;

\entry
${ad \equiv bd \pmod{nd}} \Leftrightarrow {a\equiv b \pmod{n}}$;

\entry
${b = a^{-1} \pmod{n}} \Leftrightarrow {ab \equiv 1 \pmod{n}}$;

\entry
${n \perp m} \implies
  {\left( a\equiv b \pmod{n} \land a \equiv b \pmod{m} \Leftrightarrow
  a\equiv b \pmod{nm} \right)}$ (ugogólnia się na dow. liczbę parami wzgl.
  pierwszych modułów $n_1,\ldots,n_k$);

\entry
$a \perp n \Rightarrow a^x = a^{x \mod \phi(n)} \mod n$;

% Chińskie twierdzenie o resztach.
\framebox{\vbox{
  \entry
    CRT: ${n = n_1\dots n_k}, {n_i \perp n_j}$, to
      $\forall_{a_1, \dots, a_k}\exists a\in\set{0,\dots,n-1}$ t., że
      $a\equiv a_i \pmod{n_i}$ dla $i=1,\dots,k$;

    % Przykład użycia chińskiego twierdzenia o resztach.
    % Źródło: wazniak.mimuw.edu.pl:Matematyka dyskretna 1: Wykład 11
  \entry
    \textsubentry{1}
    $x \equiv_{3} 2$, $x \equiv_{10} 7$, $x \equiv_{11} 10$,
      $x \equiv_{7} 1$;
      \textsubentry{2}
      Warunki CRT:
      $\forall_{n_i, n_j, n_i \neq n_j} NWD(n_i, n_j) = 1$
      \textsubentry{3}$n = 3\cdot 7\cdot 10\cdot 11 = 2310$,
      $m_1 = \frac{2310}{3} = 770$,
      $m_2 = \ldots = 231$,
      $m_3 = \ldots = 210$,
      $m_4 = \ldots = 330$;
      \textsubentry{4}$m_1^{-1} \mod n_1 = 1$, $\ldots$ oraz
      $x_1 = m_1 (m_1^{-1} \mod n_1)$, $\ldots$
      \textsubentry{5} R. alg. Eukl.:
      $NWD(3, 770) = 1 = 257\cdot 3 - 1\cdot 770 \Rightarrow
        x_1=-1\equiv_{3} 2$,
      $NWD(10, 231) = \ldots \Rightarrow x_2=1$,
      $NWD(11, 210) = \ldots \Rightarrow x_3 = 1$,
      $NWD(7, 330) = \ldots \Rightarrow x_4 = 1$;
      \textsubentry{6}
      $x = 2\cdot 2 \cdot 770 + 7\cdot 1\cdot 231 + 10\cdot 1\cdot 210 +
      1\cdot 1\cdot 330$;

  \entry
      Ogólniej: $a^{\phi(p_1^{\alpha_1})} \equiv 1 \pmod{p_1^{\alpha_1}}$,
      $a^{\phi(p_2^{\alpha_2})} \equiv 1 \pmod{p_2^{\alpha_2}} \Rightarrow
      a^{NWW(\phi(p_1^{\alpha_1}),\phi(p_2^{\alpha_2}))} \equiv 1 \pmod{n}$,
      gdzie $n=p_1^{\alpha_1}p_2^{\alpha_2}$;

  \entry
    $x^{118} \equiv 113 \pmod{1001}$;
    \textsubentry{1}
    $x^{118} \equiv_7 1$, $\ldots\equiv_{11} 3$, $\ldots\equiv_{13} 9$ są $\perp$,
      więc tw. Eulera;
    \textsubentry{2}
    $x^2 \equiv_7 1^{-1}$, $x^2 \equiv_{11} 3^{-1}$, $x^2 \equiv_{13} 9^{-1}$
    \textsubentry{3}
    $x^2 \equiv_7 1$, $x^2 \equiv_{11} 4$, $x^2 \equiv_{13} 3$
    \textsubentry{4}
    $x \equiv_7 \pm 1$, $x \equiv_{11} \pm 2$, $x \equiv_{13} \pm 4$
    \textsubentry{5}
    Teraz z CRT: $\ldots$: $c_1 = 5 \cdot 143$, $c_2 = 4\cdot 91$,
      $c_3 = 12\cdot 77$;
}}

% Małe twierdzenie Fermata.
\entry
MTF: ${p\in \mathbb{P} \land p\not | a} \Rightarrow a^{p-1} \equiv 1 \pmod{p}$,
  inaczej $a^p \equiv a \pmod{p}$;

% Funkcja Eulera.
\entry
F. Eulera: $\phi: \mathbb{N} \rightarrow \mathbb{N}$ t., że
  $\phi(n)=|\mathbb{Z}^*_n|=|\set{1\leq k\leq n : k \perp n}|$;

\entry
$p \in \mathbb{P} \Rightarrow \phi(p^k)=p^k-p^{k-1}$;

\entry
$m\perp n \Rightarrow \phi(mn) = \phi(m)\phi(n)$, np.:
  $\phi(2016) = \phi(2^3)\phi(3^2)\phi(7) =
  2016(1-\frac{1}{2})(1-\frac{1}{3})(1-\frac{1}{7})$;

\entry
$\phi(n) = n\prod_{\mathbb{P} \ni p | n}(1- 1/p)$;

\entry
$\sum_{d|m}\phi(d) = m$;

% Twierdzenie Eulera.
\entry
Tw. Eulera: $a\perp n \Rightarrow a^{\phi(n)} \equiv 1 \pmod{n}$;

% Twierdzenie Wilsona.
\entry
Tw. Wilsona: $p\in\mathbb{P} \Leftrightarrow (p-1)! \equiv -1 \pmod{p}$;

% RSA.
\entry
Alg. RSA: $n=pq, e, M, e \perp \phi(n)$. Wtedy $E(M) = M^e \mod n$,
  $d = e^{-1} \mod \phi(n)$, $D(C) = C^d \mod n$. Działa, bo
  $(M^e)^d \mod n = \text{z tw. Eulera} = M^{ed \mod n} = M^1 \mod n$;

% Test Millera-Rabina.
\entry
Test Millera-Rabina: $\exists_{0<a<n}a^{n-1}\not\equiv 1 \pmod{n} \Rightarrow
  n \not\in \mathbb{P}$ (słaby, bo liczby Carmichaela);

% TODO: Opis liczb Carmichaela.

% TODO: Faktoryzajca metodą Fermata.

% TODO: Faktoryzacja metodą "p-1" Pollarda.

\section{Teoria grup}

$G = \angles{e, \cdot}$; łączność, elem. neutr., odwrotność, przemienność*;

Podgrupa grupy $G$, to podzb. zamkn. na $\cdot$ i $^{-1}$;

Podgr. generowana przez $A$, to przecięcie wszystkich podgr. $G \supseteq A$;

$G(A)$ składa się ze skończonych iloczynów $g_1\dots g_k$,
  $g_i\in A \lor g_i^{-1} \in A$;

Grupa cykliczna, czyli gen. przez 1 elem.;

% Twierdzenie (Lagrange).
Tw. (Lagrange): Rząd podgr. $H$ gr. skończonej $G$ dzieli rząd $G$;

$G$ cykliczna ma $\phi(n)$ gen;

Tw.: $s(d)$, to l. elem. rzędu $d$ w $G$. $s(d) =$ if $d \not{|} n$ then $0$
  else $\phi(d)$;

% Twierdzenie (Cayley).
Tw. (Cayley): $S(Y)$, to zbiór bijekcji $Y \rightarrow Y$ ze składaniem
  ($S_n$ dla $Y=\set{1,\dots,n}$). Każda gr. rzędu $n$ jest izomorf. z pewną
  podgr.  $S_n$;

% TODO: Dokończyć teorię grup. Slajd 216.


\end{document}
